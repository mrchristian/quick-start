\documentclass{article}

\usepackage{hyperref}
\usepackage{caption}
                
\usepackage[backend=biber,hyperref=false,citestyle=authoryear,bibstyle=authoryear]{biblatex}
                
\bibliography{bibliography}
            
\usepackage{graphicx}
                
\usepackage{calc}
                
\newlength{\imgwidth}
                
\newcommand\scaledgraphics[2]{%
                
\settowidth{\imgwidth}{\includegraphics{#1}}%
                
\setlength{\imgwidth}{\minof{\imgwidth}{#2\textwidth}}%
                
\includegraphics[width=\imgwidth,height=\textheight,keepaspectratio]{#1}%
                
}
            
\begin{document}

\title{Step 1: Make a Repository}

\maketitle


A repository is the data storage location of your outputted publication.


\subsection{About}\label{H4018151}



The reposoitories use \href{https://git-scm.com/}{Git}\footnote{Git is open-source software that both \href{https://github.com/}{GitHub} and \href{https://about.gitlab.com/}{GitLab} are built on – think of it as a time machine for code and all that could do.} technology which allows for versioning of your publication.


We save to GitHub and GitLab \autocite{PerkelJeffrey2016}. GitLab can be used as \href{GitLab.com}{GitLab.com} or as a self-hosted instance for private publications, or for staging publicatiobs. We use GitLab Community Edition for self-hosting which is open-source software. GitHub is not open-source but is useful for distribution and visibility.

\begin{figure}
\scaledgraphics{33ede307-f4a4-4660-92d2-92ccb1ef61d5.jpg}{0.5}
\caption*{Git logo}\label{F22345641}
\end{figure}

\begin{figure}
\scaledgraphics{5656f6ad-7ef0-4a57-9d81-9fc2ecfc77bc.png}{0.5}
\caption*{Octocat: GitHub's mascot}\label{F44428261}
\end{figure}

\begin{figure}
\scaledgraphics{6c70c3fd-31ab-446b-bfe3-5a7ee1e2f98f.png}{0.5}
\caption*{GitLab logo}\label{F88034391}
\end{figure}


\subsection{Step-by-step guide}\label{H1473428}



These instructions are for using GitHub. The principles are the same for GitLab.


These steps will allow you to create a repository to for your publication with the option for a website.

\begin{enumerate}
\item Create an account

\begin{figure}
\scaledgraphics{56369a22-3b5a-4691-aa78-51a8bd4bba37.png}{1}
\label{F22252491}
\end{figure}


Go to the website \href{https://github.com/}{GitHub.com} and create an account.


\item Use a GitHub Template Repository





\item Turn on your website


\item Connect Fidus Writer to GitHub


\item Export your publication to GitHub


\item Add a README to your repo


\item You can allow team members to export to GitHub too — this will be covered in the invite team section.


\end{enumerate}




\printbibliography[title={Bibliography}]
\end{document}
